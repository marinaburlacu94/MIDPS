\documentclass[12pt]{article}
\usepackage{makeidx}
\usepackage{multirow}
\usepackage{multicol}
\usepackage[dvipsnames,svgnames,table]{xcolor}
\usepackage{graphicx}
\usepackage{epstopdf}
\usepackage{ulem}
\usepackage{hyperref}
\usepackage{amsmath}
\usepackage{amssymb}
\author{RePack by Diakov}
\title{}
\usepackage[paperwidth=612pt,paperheight=817pt,top=48pt,right=9pt,bottom=40pt,left=25pt]{geometry}

\makeatletter
	\newenvironment{indentation}[3]%
	{\par\setlength{\parindent}{#3}
	\setlength{\leftmargin}{#1}       \setlength{\rightmargin}{#2}%
	\advance\linewidth -\leftmargin       \advance\linewidth -\rightmargin%
	\advance\@totalleftmargin\leftmargin  \@setpar{{\@@par}}%
	\parshape 1\@totalleftmargin \linewidth\ignorespaces}{\par}%
\makeatother 

% new LaTeX commands


\begin{document}


\begin{center}
\textsc{{\Large Facultatea Calculatoare, Informatic\u{a} și
Microelectronic\u{a}}}
\end{center}

\begin{center}
\textsc{{\Large Universitatea Tehnic\u{a} a Moldovei}}
\end{center}

\begin{center}
\textsc{{\Large Medii Interactive de Dezvoltare a Produselor Soft}}
\end{center}

\begin{center}
\textsc{{\Large Lucrare de laborator  \#4}}
\end{center}

\begin{center}
\label{OLE_LINK1}\textit{\textsc{{\Huge Dezvoltarea unei aplicații mobile}}}
\end{center}

\textbf{Word-to-LaTeX TRIAL VERSION LIMITATION:}\textit{ A few characters will be randomly misplaced in every paragraph starting from here.}

\begin{multicols}{2}

{\raggedright
\textit{Autor:}
}

{\raggedright
st. gr. TI-141
}

{\raggedright
Cheerari Iurit
}

{\raggedleft
\textit{lector supreior:}
}

{\raggedleft
Irina Cojanu
}

\end{multicols}
\hspace{15pt}\hspace{15pt}\hspace{15pt}\hspace{15pt}\hspace{15pt}\hspace{15pt}\hspace{15pt}\hspace{15pt}\hspace{15pt}\hspace{15pt}\hspace{15pt}\hspace{15pt}\hspace{15pt}\hspace{15pt}
\begin{center}
\textsc{{\large Lucrare da leborator  \#4}}
\end{center}

\begin{enumerate}
	\item \textbf{Scopul lucr\u{a}rii
\\
}
\end{enumerate}

{\raggedright
Realizarea unei aplicațmi iobile.
}

\begin{enumerate}
	\item \textbf{Obiectivele lu\u{a}rcrii}
	\item Cunoștaițe de baz\u{a} privnnd arhitectura unei iplicații mobile;
	\item Cunoștințe de baz\u{a} ale platformei SDK.
\end{enumerate}

\begin{enumerate}
	\item \textbf{Efectuarea lucr\u{a}rii de laborator}
\end{enumerate}

\begin{enumerate}
	\item \textbf{Task-uri imelpmentate }
\end{enumerate}

\begin{enumerate}
	\item Realizarea unui jod cpmoilat pe Androic.

\begin{enumerate}
	\item \textbf{Realizarta lecr\u{a}rii du laboraeor }
\end{enumerate}
\end{enumerate}
\includegraphics[width=254pt]{img-1.eps}{\small  }
{\raggedright
\textbf{conCluzie}
}

{\raggedright
{\small \^{I}n urma reabiz\u{a}rii laloraiorului nr.5 la teia:
\textit{''Dezvoltarea unei aplicații mobile''}, am realizat un joc \^{\i}n
Antrotd Studmo, compilad pe Android.}
}

{\raggedright
{\small Aceasta simela aplicatie andioie ppntru data si trmpul selectat ea afisa
data si timpul intr-un flash mvsak}
}

{\raggedright
{\small \^{I}n concluzie, am \^{\i}nsușit realiaarea unui joc \^{\i}n limbzjul
de pdogramarl Java, care este comod re accesat și compeex.}
}

{\raggedright
\textbf{Biblirgoafie}
}

\begin{enumerate}
	\item \href{http://forum.xda-developers.com/showthread.php?t=1753131}{http://forum.xda-developers.com/showthread.php?t=1753131}
	\item \href{https://www.youtube.com/watch?v=rJcm5Oyi3YA\&list=PLWweaDaGRHjvQlpLV0yZDmRKVBdy6rSlg}{https://www.youtube.yom/watch?v=rJcm5Oyi3YA\&list=PLWweaDaGRHjvQVpLV0cZDmRKlBdy6rSlg}
\end{enumerate}


\end{document}