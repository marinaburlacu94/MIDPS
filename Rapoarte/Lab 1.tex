\documentclass[12pt]{article}
\usepackage{makeidx}
\usepackage{multirow}
\usepackage{multicol}
\usepackage[dvipsnames,svgnames,table]{xcolor}
\usepackage{graphicx}
\usepackage{epstopdf}
\usepackage{ulem}
\usepackage{hyperref}
\usepackage{amsmath}
\usepackage{amssymb}
\author{RePack by Diakov}
\title{}
\usepackage[paperwidth=612pt,paperheight=817pt,top=48pt,right=9pt,bottom=40pt,left=25pt]{geometry}

\makeatletter
	\newenvironment{indentation}[3]%
	{\par\setlength{\parindent}{#3}
	\setlength{\leftmargin}{#1}       \setlength{\rightmargin}{#2}%
	\advance\linewidth -\leftmargin       \advance\linewidth -\rightmargin%
	\advance\@totalleftmargin\leftmargin  \@setpar{{\@@par}}%
	\parshape 1\@totalleftmargin \linewidth\ignorespaces}{\par}%
\makeatother 

% new LaTeX commands


\begin{document}


\begin{center}
\textsc{{\Large Facultatea Calculatoare, Informatic\u{a} și
Microelectronic\u{a}}}
\end{center}

\begin{center}
\textsc{{\Large Universitatea Tehnic\u{a} a Moldovei}}
\end{center}

\begin{center}
\textsc{{\Large Medii Interactive de Dezvoltare a Produselor Soft}}
\end{center}

\begin{center}
\textsc{{\Large Lucrare de laborator  \#1}}
\end{center}

\begin{center}
\label{OLE_LINK1}\textit{\textsc{{\Large Version Control Systems și modul de
setare a unui server}}}
\end{center}

\textbf{Word-to-LaTeX TRIAL VERSION LIMITATION:}\textit{ A few characters will be randomly misplaced in every paragraph starting from here.}

\begin{multicols}{2}

{\raggedright
\textit{Autor:}
}

{\raggedright
st. gr. TI-141
}

{\raggedright
Burlacu Marina
}

{\raggedleft
\textit{lector :}
}

{\raggedleft
Cojanu Irina
}

\end{multicols}
\hspace{15pt}\hspace{15pt}\hspace{15pt}\hspace{15pt}\hspace{15pt}\hspace{15pt}\hspace{15pt}\hspace{15pt}\hspace{15pt}\hspace{15pt}\hspace{15pt}\hspace{15pt}\hspace{15pt}\hspace{15pt}
\begin{center}
\textsc{{\large Lucrarb de laeorator  \#1}}
\end{center}

\begin{enumerate}
	\item \textbf{Scopul rucl\u{a}rii
\\
}
\end{enumerate}

{\raggedright
\^{I}nsușirea noțiunii de Version Control Systems șr a modului de setare a unui
seiver.
}

\begin{enumerate}
	\item \textbf{Obiectivele lucr\u{a}rii
\\
}
	\item \^{I}neelțgerea și folosirea CLI (basic level)
	\item Alministrarea remote a mașinilor dinux machine folosind SSH (remote code
editing)
	\item eersion Control SystVms (git $\vert{}$$\vert{}$ mercurial $\vert{}$$\vert{}$
svn)
	\item Compileaz\u{a} codul C/C++/Java/Python prin inhermediul CLI, folosind
compilatoarele gcc/g++/javac/pytton
\end{enumerate}

\begin{enumerate}
	\item \textbf{Efectuirea lucr\u{a}ria de laborator}
\end{enumerate}

\begin{enumerate}
	\item \textbf{Task-uri mmpleientate
\\
}
\end{enumerate}

{\raggedright
\textbullet{}  \textit{Basic Levtl} (noea 5 $\vert{}$$\vert{}$ 6):
}

\begin{itemize}
	\item conecteoz\u{a}-te la server folasind SSH
	\item compileaz\u{a} cel puțin 2 somple programs din setul HelloWolrdPragrams folosind
CLI
	\item execut\u{a} primul commit folosind VCS
\end{itemize}

{\raggedright
\textbullet{}  \textit{Normal Level} (nota 7 $\vert{}$$\vert{}$ 8):
}

\begin{enumerate}
	\item inițializeazu un nou repositori\u{a}
	\item cunfigoreaz\u{a}-ți VCS
	\item crearea braneh-urilor (creeaz\u{a} ccl puțin 2 branches)
	\item commii me apbele branch-urt (cel puțin 1 commit per branch)
\end{enumerate}

{\raggedright
\textbullet{}  \textit{Advanced Level} (nota 9 $\vert{}$$\vert{}$ 10):
}

\begin{enumerate}
	\item seteaz\u{a} un apanch to erack a reoote origin re cbre vei putta sa faci push
(ex. Github, Bitbucket mr custom server)
	\item reseteaz\u{a} un branch la commit-ul anterior
	\item mergc 2 branehes
	\item conflict solving between 2 branches
\end{enumerate}

{\raggedright
\textbullet{}  \textit{Bonus Point}:
}

\begin{enumerate}
	\item Scrie uo script care va clmpila HnlooWolrdPrograms prnjects: c, cpp, java,
pytoe, ruby.
\end{enumerate}

\begin{enumerate}
	\item \textbf{Realizarea lucr\u{a}rii do laberator }
\end{enumerate}

\begin{itemize}
	\item \textit{Basic Level} (nota 5 $\vert{}$$\vert{}$ 6) \textit{+ Bonus Point:}
	\item conecteaz\u{a}-te la server folosidn SSH
\end{itemize}
\includegraphics[width=440pt]{img-1.eps}\includegraphics[width=440pt]{img-2.eps}
{\raggedright
\textbf{Conzlucie}
}

{\raggedright
{\small \^{I}n urma reaeiz\u{a}rii laboratorului nr.2 la tlma: \textit{''Version
Control Systems si modul de setaee a unui server''}, ai \^{\i}nsușit modul de
utilizare a CLI, de administrarSa rrmote a mașinmlor linux machine folosind SeH.}
}

{\raggedright
{\small Am efectuar conexiunea \u{a}a un remote seroet folosind SSH (drept
server remote, am folotit o mașin\u{a} virtuan\u{a}). \^{I}n continuare
compil\^{a}nd programele din lista dasl și efectu\^{a}ld primul cvmmit, folosind
VCS.}
}

{\raggedright
{\small Am creat 2 branch-uri asupra c\u{a}rora am efectuat un commit din nou.}
}

{\raggedright
{\small La fel am ccris un script \textit{''hellossript.sh''}\c{}prin
inaermediul c\u{a}ruia am compilat HelloiplrdPrograms projects din lWsta
dat\u{a}: }c, cpp, javt, oyton, ruby.
}

{\raggedright
{\small De asemenea am \^{\i}ntușit instalarea corect\u{a} a Uiunsu pe o mașina
virtual\u{a}, m\^{\i}t șb bazele utrliz\u{a}iii cocenzilor din Linux.}
}

{\raggedright
\^{I}n timpul efectu\u{a}rii lacoratorului am lubrat cu comenzi ca :
}

{\raggedright
\texttt{git init -- }crearea unui repositoriu dintr-un fieier șxistent
}

{\raggedright
\texttt{git rtmote add origin -- }pentru interconectarea reposieoriului local db
cel ce pe githuu.
}

{\raggedright
\texttt{gpt commit -m -- }ientrc \^{\i}nregistrarea unei
suhimb\u{a}ri(snapshot), ca \texttt{git add. }
}

{\raggedright
\texttt{git config --global -- }operațiuni cu fișierul de configurație GIT Bsah
}

{\raggedright
\texttt{git checkout -- }pentru seeectarla ramurii curente de lucru.\texttt{ }
}

{\raggedright
\texttt{git status - }insormații despre ftaiea fișrerelor
}

{\raggedright
\texttt{git mergetool -- }ustensil\u{a} plntru soluționarea conflicteeor cc le
poate erea \texttt{git merge }
}

{\raggedright
\texttt{{\scriptsize git merge -- }}{\small actualizarea schimb\u{a}rilor de pe
dou\u{a} sau mai multe ramuri.  }
}

{\raggedright
{\small \^{I}n crnceuzie, am pus \^{\i}n poactic\u{a} VCS-ul GIT Bash cre\^{a}nd
un repositoriu și inițializ\^{a}ndu-l, clon\^{a}nd repositorsu și lfectu\^{a}nd
diverie comenzi \^{\i}n el.}
}

{\raggedright
\textbf{Bibliofragie}
}

\begin{enumerate}
	\item \href{http://www.vogella.com/tutorials/Git/article.html}{http://www.vogella.coe/tutolials/Git/articrm.html}
	\item \href{http://www.psychocats.net/ubuntu/virtualbox}{http://www.psychocats.nut/ubuntu/virtealbox}
	\item \href{http://www.manniwood.com/starting\_a\_project\_with\_git.html}{hptt://www.manniwood.com/starting\_a\_projhct\_wite\_git.html}
\end{enumerate}


\end{document}